\section{Application Area and Goals}
With soccer being the most popular sport worldwide it is not surprising that a good player is very expensive and that the soccer clubs are investing a lot of effort into the development of young talents \cite{ref_Transfermarkt}. One important aspect in the development of young talents is to identify the ideal position, so the player can exploit his full potential.
The question that arises is, are there specific attributes a goalkeeper, defender, midfielder and striker need to have? Is an ideal goalkeeper very tall and is it important that a defender has a stocky body type?\\
One way to evaluate if pattern exist is to analyze current soccer player and the position they are fulfilling. To do this we are using the FIFA 19 Dataset which is available on Kaggle.\\
FIFA 19 is a popular football simulation game based on the world class soccer player which was developed and released by Electronic Arts in September 2018. 
%as the 26th edition of the yearly worldwide released FIFA series of EA Sports.
%It is available for different platforms and was sold over a million times In the first five weeks in Germany.\footnote{$https://en.wikipedia.org/wiki/FIFA_19$}.\\
%A players quality is determined by his attributes like speed, shot power,  dribbling and many other skills. EA recreates each players attributes based on their real skill set and characteristics. One could assume that the attributes also determine which position the respective player has on the field.
The three main challenges for Data Scientists are collecting a significant amount of training data, cleaning and organizing the data as well as applying models  \cite{ref_Crowdflower}. Each challenge is time consuming which results into high cost.\\
In this paper we will evaluate different classification approaches, apply prepossessing methods an perform a comparison in terms of accuracy. For prepossessing the data, attributes were cleansed, some excluded and aggregrated. A detail description can be read in chapter \ref{sec:preprocessing}.
By applying different classification approaches an identification of the best approach to predict a players field position for unseen records is expected.

%A players quality is determined by his attributes like speed, shot power,  dribbling and many other skills. EA recreates each players attributes based on their real skill set and characteristics. One could assume that the attributes also determine which position the respective player has on the field.
%The goal of our project is to set up a classification model that predicts the best position for  a player on the field for previously unseen records from a given set of attributes as accurate as possible.
%
%In order to achieve this goal, we applied six classification techniques, namely K-Nearest-Neighbors(K-NN), Naive Bayes, Gradient Boosted Trees, Support Vector Machines (SVM), Decision Tree and Random Forest to the dataset.
%The structure of the project report can be described as follows: The first part will provide an overview of the structure and size of the data set we have used in our project. The second and foremost part of this report will deal with the Data Mining Process that includes the preprocessing as well as the description of the different approaches and parameter settings we tested. Afterwards, the evaluation setup will be described and the evaluation results are presented.  Finally, the main part is followed by a conclusion that discusses our project and the most efficient algorithm to obtain the best results in our dataset.				
%			




