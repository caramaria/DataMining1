
\subsection{K-Nearest Neighbors (KNN)}
\label{sec:KNN}
K-Nearest Neighbors classifies unseen cases based on the k nearest neighbors. 
%As input to train our model FIFA 2019 data set is used. 
First, the ``optimize parameters" operator was applied for values of k between 1 and 100 as well as a 10-fold cross validation. The result of this optimization showed that the ideal value for k = 62 which provides an accuracy of 87.83\%. The best performing class was \textit{goalkeeper} with a precision of 100\% and the worst performing class was midfielder with a precision of 81.45\%. \\
Applying our model to the FIFA17 test data set the following results could be measured:\\
\begin{table}[H]
\label{Tab:knn}
\centering
\begin{tabular}{@{}lll@{}}
\toprule
                   & Training data & Test data \\ \midrule
Accuracy           & 87.83\%       & 87.37\%   \\
Weighted recall    & 88.46\%       & 87.92\%   \\
Weighted precision & 90.08\%       & 88.86\%   \\ \bottomrule
\end{tabular}
\caption{Results from KNN algorithm on training and test data}
\end{table}
In phase two of the mining process using KNN we started to optimize the algorithm by optimizing the attributes.
To do this we used the ``Optimize Selection'' and the ``Optimize Selection (Evolutionary)'' operator which determined the weight for the attributes based on the Gini Index and the Information Gain.
The optimal attributes are presented in table 4.
\begin{table}[H]
\begin{tabular}{p{3.5cm}|p{7.5cm}l|l}
\hline 
Optimize Selection (9) & Crossing, Finishing, HeadingAccuracy, ShortPassing, Dribbling, LongPassing, Vision, Marking, SlidingTackle\\
\hline
Information Gain (24)& Crossing, Finishing, HeadingAccuracy, ShortPassing, Volleys, Dribbling, FKAccuracy, LongPassing, BallControl, Acceleration, SprintSpeed, Agility, Balance, Jumping, Stamina, Strength, LongShots, Interceptions, Positioning, Composure, Marking, StandingTackle, SlidingTackle, Vision \\
\hline 
Gini Index (21) & Crossing, Finishing, HeadingAccuracy, ShortPassing, LongPassing, BallControl, Acceleration, SprintSpeed, Agility, Balance, Jumping, Stamina, Strength, LongShots, Interceptions, Positioning, Composure, Marking, StandingTackle, SlidingTackle, Vision\\ \hline
\end{tabular}
\label{Tab:knn2}
\caption{Selection of attributes}
\end{table}	
By selecting the attributes, the accuracy increased to 88.28 \% (Optimize Selection), 88.15 \% (Information Gain), 87.97\% (Gini Index). Table 5 shows the results on FIFA 2017 data set after the optimize selection.

\begin{table}[H]
\begin{tabular}{@{}llll@{}}
\toprule
                                        & Accuracy & Weighted Recall & Weighted Precision \\ \midrule
\multicolumn{1}{l|}{Optimize Selection} & 87.24\%  & 87.84\%         & 88.56\%            \\
\multicolumn{1}{l|}{Information Gain}          & 87.39\%  & 87.94\%         & 88.95\%            \\
\multicolumn{1}{l|}{Gini Index}         & 87.18\%  & 87.70\%         & 88.65\%            \\ \bottomrule
\end{tabular}
\label{Tab:KNNResults}
\caption{Results of KNN on test data}
\end{table}

As last optimization the ``Optimize Parameters" operator was used again, but only for the selected attributes with k values (1-100, 100 steps). This resulted to a k = 57 with the subset of attributes from the operator, and a k = 1 for both Information Gain and Gini Index. With this last optimization the test results couldn't be improved anymore and the accuracy dropped by 0.2\%.
